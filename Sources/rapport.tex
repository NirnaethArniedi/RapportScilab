\documentclass[12pt,titlepage=true]{article}

\usepackage[utf8x]{inputenc}

\usepackage[frenchb]{babel}
\usepackage{amsmath}
\usepackage{amssymb}
\usepackage{amsfonts}
\usepackage{appendix}
\usepackage{array}
\usepackage{bigdelim} % comme son nom l'indique cf plaquette sponsor AvironX
\usepackage{bold-extra}
\usepackage{color}
\usepackage{colortbl} %tableaux colorés, cf plaquette sponsor AvironX
\usepackage{caption}
\usepackage{easytable}
\usepackage{epsfig}
\usepackage{etex}
\usepackage{etoolbox}
\usepackage{fancyhdr} % en tête et pieds de page
\usepackage{graphicx}
\usepackage{here}
\usepackage{layout} % \layout au début de doc -> longueurs caractéristiques de la mise en page
\usepackage{makeidx}
\usepackage{mathrsfs}
\usepackage{mathtools}
\usepackage{mdframed} %\begin{mdframed} _-> encadrement qui est beaucoup plus souple que fbox, gère les saut de page etc...
\usepackage{multirow}
\usepackage{nopageno}
\usepackage{pslatex}
\usepackage{pstricks-add}
\usepackage[retainorgcmds]{IEEEtrantools} % des beaux tableaux (équations, matrice etc...) cf rapport TIPE
\usepackage[Gray, squaren]{SIunits} %éponyme, cf TIPE
\usepackage{stmaryrd} %pour les intervalles d'entiers \llbracket ou \rrbracket
\usepackage{shorttoc} %2e table des matières appelée par \shorttableofcontents{titre}{profondeur} cf rapport Stage FHM
\usepackage{textcomp} %polices
\usepackage[titles]{tocloft}
\usepackage{titlesec}
\usepackage{txfonts}
\usepackage{ucs}
\usepackage{wrapfig} % comme son nom l'indique cf rapTIPE
\usepackage{tocbasic}
\usepackage{lmodern}
%\usepackage[a4paper]{geometry}
%\usepackage{pdfpages}
%\usepackage[round]{natbib}
%\FrenchFootnotes
%\AddThinSpaceBeforeFootnotes
%\bibliographystyle{unsrt-fr}

\usepackage{xcolor}
\usepackage{dsfont}

\definecolor{bleuX}{RGB}{0,62,92}

\renewcommand{\arraystretch}{1.3}
\renewcommand\labelitemi{\textbullet}
\renewcommand{\theequation}{\arabic{subsection}.\arabic{equation}}

\input{input/meptex}

%\usepackage{layout}
%\usepackage{rowcolor}

\newcommand{\esp}{\mathbb{E}}
\renewcommand{\exp}{\mathrm{e}^}
\renewcommand{\P}{\mathbb{P}}
\newcommand{\de}{\mathrm{d}}
\newcommand{\R}{\mathbb{R}}
\newcommand{\Xt}{\tilde{X}}
\newcommand{\St}{\tilde{S_n}}
\newcommand{\sigt}{\tilde{\sigma}}


\title{Vitesse d'invasion pour un modèle de reproduction et dispersion}
\author{Augustin Lenormand \and Basile Bruneau}

\begin{document}
	\maketitle

	\section{Grandes déviations d'une marche aléatoire}
		\subsection{}\setcounter{equation}{0} %Question 1
			
			\renewcommand\labelitemi{\textbullet}
			
			\begin{itemize}
	
				\item	Soit $\alpha\in [0,1]$ et $\lambda_1, \lambda_2$ dans $\mathbb{R}$.
	
						\begin{IEEEeqnarray*}{l C c}
							\esp( \exp{(\alpha \lambda_1 + (1 - \alpha)\lambda_2) X}) &     =     &  \esp((\exp{\lambda_1 X})^{\alpha}(\exp{\lambda_2 X})^{1-\alpha}\\
																   					  & \leqslant & (\esp(\exp{\lambda_1 X}))^{\alpha} (\esp(\exp{\lambda_2 X}))^{1- \alpha}
						\end{IEEEeqnarray*}
	
	
						Ici la dernière inégalité est l'inégalité de Hölder. En passant au logarithme il vient donc naturellement :	
	
						\begin{equation*}
							\Lambda((\alpha \lambda_1 + (1 - \alpha)\lambda_2) X) \leqslant \alpha \Lambda(\lambda_1 X) + (1-\alpha) \Lambda(\lambda_2 X)
						\end{equation*}
		
						\fbox{Donc $\Lambda$ est convexe.}
		
				\item	De même les fonctions $f_\lambda$ telles que $f_\lambda (x)=\lambda x - \Lambda (\lambda)$ sont toutes convexes. Donc par définition leurs épigraphes sont convexes.
			
						Or l'épigraphe du supremum pour $\lambda$ dans $\mathbb{R}$ de ces fonctions est l'intersection des épigraphes de toutes ces fonctions. Comme intersection d'ensemble convexes , il est donc convexe lui aussi. Donc l'épigraphe de $\Psi$ est convexe.
		
						\fbox{Donc $\Psi$ est convexe.}

				\item	$\Lambda(0)=0$ donc :
							
						\begin{equation*}
							\forall x \in \mathbb{R}, \sup_{\lambda \in \mathbb{R}}(\lambda x - \Lambda(\lambda)) \geqslant 0\cdotp x - \Lambda(0) = 0
						\end{equation*}	
						
						\underline{Donc $\Psi\geqslant0$.}
			
						Soit $\lambda$ dans $\mathbb{R}$.
		
						\begin{equation*}
							\exp{f_\lambda(m))}=\frac{\exp{\lambda \esp(X)}}{\esp(\exp{\lambda X})}
						\end{equation*}			
			
						Or la fonction $x \mapsto \exp{\lambda x}$ est convexe, donc d'après l'inégalité de Jensen,	$\exp{\lambda \esp(X)}\leqslant\esp(\exp{\lambda X})$. Ainsi $\exp{f_\lambda(m))}\leqslant 1$ et $\lambda \esp(X) - \Lambda(\lambda)\leqslant 0$. 
			
						Donc $\Psi(\esp(X))\leqslant0$. Or $\Psi\geqslant0$.
			
						\fbox{Donc $\Psi$ admet un minimum en $m$ et $\Psi(m)=0$.}

				\item	Soit $x\geqslant m$ et $\lambda <0$. Alors on peut écrire :
						
						\begin{IEEEeqnarray*}{C}
							\lambda x -\Lambda(\lambda) \leqslant \lambda m - \Lambda(\lambda) = 0\\
							\lambda x -\Lambda(\lambda) \leqslant0 \leqslant\sup_{\lambda \in \mathbb{R}}\{\lambda x - \Lambda(\lambda)\}
						\end{IEEEeqnarray*}
			
						\fbox{Donc prendre le supremum des $f_\lambda$ pour $\lambda\geqslant0$ est suffisant pour définir $\Psi$}
		
			\end{itemize}
		 
		\subsection{}\setcounter{equation}{0} %Question 2
	
			\begin{IEEEeqnarray*}{cCc}
				\P(S_n \geqslant nx) & = & \esp(\mathds{1}_{S_n-nx\geqslant0})
			\end{IEEEeqnarray*}
	
			Comme $S_n-nx\geqslant0$ alors pour tout $\lambda$ positif, $\exp{\lambda (S_n-nx)}\geqslant1$ et

			\begin{equation*}
				\esp(\mathds{1}_{S_n-nx\geqslant0}) \leqslant \esp(\exp{\lambda (S_n-nx)}\mathds{1}_{S_n-nx\geqslant0}) \leqslant \esp(\exp{\lambda (S_n - x)})
			\end{equation*}
	
			On a donc :
			
			\begin{equation*}
				\P(S_n \geqslant nx) \leqslant \exp{-\lambda n x}\esp(\exp{\lambda S_n})
			\end{equation*}
			
			Comme les variables $(X_i)_{i\in\llbracket 1,n\rrbracket}$ sont des v.a indépendantes de même loi on peut écrire que $\esp(\exp{\lambda S_n})=(\esp(\exp{\lambda X}))^n=\exp{n\Lambda(\lambda)}$.
	
			Par le calcul on obtient alors :
			\begin{IEEEeqnarray*}{rCl}
				\P(S_n \geqslant nx)      	   		  & \leqslant &  \exp{-\lambda n x}\exp{n\Lambda(\lambda)} 				 \\
				\log \P(S_n \geqslant nx) 	   		  & \leqslant & -\lambda n x + n\Lambda(\lambda) 						 \\
		-		\frac{1}{n} \log \P(S_n \geqslant nx) & \geqslant & \lambda  x -\Lambda(\lambda) \IEEEyesnumber \label{IneQ2}\\
			\end{IEEEeqnarray*}
	
			\ref{IneQ2} est vrai pour tout $\lambda$ positif et pour tout $n$. Donc le passage au supremum pour $\lambda$ positif et à la limite inférieure pour $n$ est possible et ne modifie pas l'inégalité.
	
			On obtient bien alors :
	
			\begin{equation}
				\boxed{\liminf_{n\rightarrow\infty}-\frac{1}{n} \log \P(S_n \geqslant nx)\geqslant\sup_{\lambda\geqslant0}\{\lambda x - \Lambda(\lambda)\}}\label{resQ2}
			\end{equation}
	
		\subsection{}\setcounter{equation}{0} %Question 3
	
			\begin{itemize}
				\item 	Montrons tout d'abord l'égalité proposée par l'énoncée.	Soit $\Phi$ une fonction mesurable bornée. Alors, en prenant $\varPi $ la loi de probabilité du vecteur $(X_1,\cdots,X_n)$,
				
						\begin{equation}
							\esp(\Phi(X_1,\cdots,X_n))=\int_R\cdots\int_R \Phi(x_1,\cdots,x_n)\varPi(\de x_1, \cdots, \de x_n) \label{eq1-Q3}
						\end{equation}
				
						Or les v.a. $(X_i)_{i\in\llbracket 1,n\rrbracket}$ sont i.i.d. donc si on note $\P_X$ leur loi de probabilité, on peut écrire $\varPi(\de x_1, \cdots, \de x_n)=\prod\limits_{i=0}^{n}\P_X(\de x_i)=\prod\limits_{i=0}^{n}\P(X\in\de x_i)$.
				
						On peut alors réécrire \ref{eq1-Q3} et utiliser la relation fournie par l'énoncé ainsi :
				
						\begin{IEEEeqnarray*}{cCl}
							\esp(\Phi(X_1,\cdots,X_n)) & = & \int_R\cdots\int_R\Phi(x_1,\cdots,x_n)\prod\limits_{i=0}^{n}\P(X\in\de x_i)\\
													   & = & \int_R\cdots\int_R \Phi(x_1,\cdots,x_n)\prod\limits_{i=0}^{n}\left(\frac{\esp(\exp{\tau X})}{\exp{\tau x_i}}\P(\Xt\in \de x_i)\right)													 \\
													   & = & \esp(\exp{\tau X})^n \int_R\cdots\int_R \Phi(x_1,\cdots,x_n) \exp{-\tau\tilde{S_n}} \prod\limits_{i=0}^{n}\P(\Xt\in \de x_i)										  \\
													   & = & \esp(\exp{\tau X})^n \int_R\cdots\int_R \Phi(x_1,\cdots,x_n)\exp{-\tau\tilde{S_n}} \tilde{\varPi}(\de x_1, \cdots, \de x_n)																   \\
						\end{IEEEeqnarray*}
				
						Ici, $\tilde{S_n}=\sum\limits_{i=0}^{n}\Xt_i$ et $\tilde{\varPi}$ est la loi de probabilité du vecteur $(\Xt_1,\cdots,\Xt_n)$ car les $(\Xt_i)_{i\in\llbracket 1,n\rrbracket}$ sont aussi i.i.d. et donc $\prod\limits_{i=0}^{n}\P(\Xt\in \de x_i)=\tilde{\varPi}(\de x_1, \cdots, \de x_n)$ de la même manière que pour les $X_i$.
			
						On reconnaît alors donc bien l'expression souhaitée :
				
						\begin{equation}
							\boxed{\esp(\Phi(X_1,\cdots,X_n))=\esp(\exp{\tau X})^n\esp\left(\Phi(\Xt_1,\cdots,\Xt_n)\exp{-\tau\tilde{S_n}}\right)} \label{eq2-Q3}
						\end{equation}
			
				\item	On calcule l’espérance de $\Xt$ ce qui sera utile plus bas.
						
						\begin{IEEEeqnarray*}{rL}
							\esp(\Xt)	 		 & = \int_{\R}u\,\P(\Xt\in\de u)                                     \\
									 	 		 & = \int_{\R}\frac{u \exp{\tau u}}{\esp(\exp{\tau X})} \P(X\in\de u)\\
							 	 	 	 		 & = \frac{\esp(X \exp{\tau X})}{\esp(\exp{\tau X})}                 \\
							\Aboxed{\esp(\Xt)    & = x \IEEEyesnumber \label{eq3-Q3}  }                              \\
						\end{IEEEeqnarray*}
				\\
				\\
				
				
				\item	Grâce à \ref{eq2-Q3} et \ref{eq3-Q3} on peut alors écrire :
				
						\begin{IEEEeqnarray*}{cCl}
							\P(nx\leqslant S_n\leqslant ny) &     = 	& \esp(\mathds{1}_{S_n\in[nx,ny]}) 											 \\
															&     = 	& \esp(\exp{\tau X})^n \,\esp\left(\exp{-\tau\tilde{S_n}}\mathds{1}_{\tilde{S_n}\in[nx,ny]} \right)																		 			 \\
															& \geqslant & \esp(\exp{\tau X})^n \exp{-\tau ny}\esp(\mathds{1}_{\tilde{S_n}\in[nx,ny]})\\	
						\end{IEEEeqnarray*}
				
						On pose $\mathrm{Var}(\Xt)=\tilde{\sigma}$. On peut alors écrire que :
				
						\begin{equation*}
							\esp\left(\mathds{1}_{\tilde{S_n}\in[nx,ny]}\right)=\esp\left(\mathds{1}_{\theta_n\in\left[0,\frac{\sqrt{n}}{\sigt}(y-x)\right]}\right) , \theta_n=\frac{\sqrt{n}}{\sigt}\left(\frac{\tilde{S_n}}{n}-x\right)
						\end{equation*}
				
						Or $\esp(\Xt)=x$, donc pour $n$ grand $\theta_n\sim\mathcal{N}(0,1)$ et $\frac{\sqrt{n}}{\sigt}(y-x)\rightarrow\infty$. En limite, on peut donc écrire :
				
						\begin{equation*}
							\esp\left(\mathds{1}_{\tilde{S_n}\in[nx,ny]}\right) \underset{n\to+\infty}{\longrightarrow}\int\limits_{0}^{\infty}\frac{1}{\sqrt{2\pi}}\exp{\frac{-x^2}{2}}\de x = 1/2.
						\end{equation*}
				
						Donc pour $n$ suffisamment grand il existe $1/2>\varepsilon>0$ tel que :
				
						\begin{IEEEeqnarray*}{cCl}
							\log \P(nx\leqslant S_n\leqslant ny) 			 & \geqslant & n (\Lambda (\tau) -\tau y  ) + \log (1/2-\varepsilon)\\
							-\frac{1}{n}\log \P(nx\leqslant S_n\leqslant ny) & \leqslant & \tau y - \Lambda (\tau) - \underbrace{\frac{\log (1/2-\varepsilon)}{n}}_{\underset{n\to\infty}{\longrightarrow \,0}}                           \\
						\end{IEEEeqnarray*}
				
						Un passage à la limite suffit alors à montrer que :
				
						\begin{equation}
							\boxed{\limsup_{n\to\infty}-\frac{1}{n}\log \P(nx\leqslant S_n\leqslant ny) \leqslant \tau y - \Lambda (\tau)}\label{resQ3}
						\end{equation}
				
			\end{itemize}
	
	\subsection{}\setcounter{equation}{0}
		
		
		\begin{IEEEeqnarray*}{RCL}
			\P(S_n \geqslant nx) 		 		     & \geqslant & \P(nx\leqslant S_n\leqslant ny) 					 \\
			\log \P(S_n \geqslant nx) 				 & \geqslant & \log \P(nx\leqslant S_n\leqslant ny) 			 \\
			-\frac{1}{n} \log \P(S_n \geqslant nx)   & \leqslant & -\frac{1}{n} \log \P(nx\leqslant S_n\leqslant ny) \\
			\limsup_{n\to\infty}-\frac{1}{n} \log \P(S_n \geqslant nx) & \leqslant & \limsup_{n\to\infty} -\frac{1}{n} \log \P(nx\leqslant S_n\leqslant ny) \IEEEyesnumber \label{eq1-Q4}\\
		\end{IEEEeqnarray*}
		
		En combinant les inégalités \ref{resQ2}, \ref{resQ3} et \ref{eq1-Q4} on obtient que pour tout $m\leqslant x <y$ :
		
		\begin{equation*}
			\Psi(x) \leqslant \liminf_{n\rightarrow\infty}-\frac{1}{n} \log \P(S_n \geqslant nx) \leqslant 	\limsup_{n\to\infty}-\frac{1}{n} \log \P(S_n \geqslant nx) \leqslant \tau y - \Lambda (\tau)
		\end{equation*}
		
		On peut alors faire tendre $y$ vers $x$ et utiliser la majoration $\tau x - \Lambda(\tau) \leqslant \Psi(x) $ pour obtenir : 
		
		\begin{equation*}
			\Psi(x) \leqslant \liminf_{n\rightarrow\infty}-\frac{1}{n} \log \P(S_n \geqslant nx) \leqslant 	\limsup_{n\to\infty}-\frac{1}{n} \log \P(S_n \geqslant nx) \leqslant \Psi(x) 
		\end{equation*}
		
		Ce qui démontre bien l'identité voulue :
		
		\begin{equation}
			\boxed{\lim_{n\to\infty}-\frac{1}{n} \log \P(S_n \geqslant nx) = \Psi(x)} \label{resQ4}
		\end{equation}
	
\end{document}